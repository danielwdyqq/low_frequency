
%% bare_conf.tex
%% V1.3
%% 2007/01/11
%% by Michael Shell
%% See:
%% http://www.michaelshell.org/
%% for current contact information.
%%
%% This is a skeleton file demonstrating the use of IEEEtran.cls
%% (requires IEEEtran.cls version 1.7 or later) with an IEEE conference paper.
%%
%% Support sites:
%% http://www.michaelshell.org/tex/ieeetran/
%% http://www.ctan.org/tex-archive/macros/latex/contrib/IEEEtran/
%% and
%% http://www.ieee.org/

%%*************************************************************************
%% Legal Notice:
%% This code is offered as-is without any warranty either expressed or
%% implied; without even the implied warranty of MERCHANTABILITY or
%% FITNESS FOR A PARTICULAR PURPOSE! 
%% User assumes all risk.
%% In no event shall IEEE or any contributor to this code be liable for
%% any damages or losses, including, but not limited to, incidental,
%% consequential, or any other damages, resulting from the use or misuse
%% of any information contained here.
%%
%% All comments are the opinions of their respective authors and are not
%% necessarily endorsed by the IEEE.
%%
%% This work is distributed under the LaTeX Project Public License (LPPL)
%% ( http://www.latex-project.org/ ) version 1.3, and may be freely used,
%% distributed and modified. A copy of the LPPL, version 1.3, is included
%% in the base LaTeX documentation of all distributions of LaTeX released
%% 2003/12/01 or later.
%% Retain all contribution notices and credits.
%% ** Modified files should be clearly indicated as such, including  **
%% ** renaming them and changing author support contact information. **
%%
%% File list of work: IEEEtran.cls, IEEEtran_HOWTO.pdf, bare_adv.tex,
%%                    bare_conf.tex, bare_jrnl.tex, bare_jrnl_compsoc.tex
%%*************************************************************************

% *** Authors should verify (and, if needed, correct) their LaTeX system  ***
% *** with the testflow diagnostic prior to trusting their LaTeX platform ***
% *** with production work. IEEE's font choices can trigger bugs that do  ***
% *** not appear when using other class files.                            ***
% The testflow support page is at:
% http://www.michaelshell.org/tex/testflow/



% Note that the a4paper option is mainly intended so that authors in
% countries using A4 can easily print to A4 and see how their papers will
% look in print - the typesetting of the document will not typically be
% affected with changes in paper size (but the bottom and side margins will).
% Use the testflow package mentioned above to verify correct handling of
% both paper sizes by the user's LaTeX system.
%
% Also note that the "draftcls" or "draftclsnofoot", not "draft", option
% should be used if it is desired that the figures are to be displayed in
% draft mode.
%
\documentclass[conference]{IEEEtran}
% Add the compsoc option for Computer Society conferences.
%
% If IEEEtran.cls has not been installed into the LaTeX system files,
% manually specify the path to it like:
% \documentclass[conference]{../sty/IEEEtran}





% Some very useful LaTeX packages include:
% (uncomment the ones you want to load)


% *** MISC UTILITY PACKAGES ***
%
%\usepackage{ifpdf}
% Heiko Oberdiek's ifpdf.sty is very useful if you need conditional
% compilation based on whether the output is pdf or dvi.
% usage:
% \ifpdf
%   % pdf code
% \else
%   % dvi code
% \fi
% The latest version of ifpdf.sty can be obtained from:
% http://www.ctan.org/tex-archive/macros/latex/contrib/oberdiek/
% Also, note that IEEEtran.cls V1.7 and later provides a builtin
% \ifCLASSINFOpdf conditional that works the same way.
% When switching from latex to pdflatex and vice-versa, the compiler may
% have to be run twice to clear warning/error messages.






% *** CITATION PACKAGES ***
%
%\usepackage{cite}
% cite.sty was written by Donald Arseneau
% V1.6 and later of IEEEtran pre-defines the format of the cite.sty package
% \cite{} output to follow that of IEEE. Loading the cite package will
% result in citation numbers being automatically sorted and properly
% "compressed/ranged". e.g., [1], [9], [2], [7], [5], [6] without using
% cite.sty will become [1], [2], [5]--[7], [9] using cite.sty. cite.sty's
% \cite will automatically add leading space, if needed. Use cite.sty's
% noadjust option (cite.sty V3.8 and later) if you want to turn this off.
% cite.sty is already installed on most LaTeX systems. Be sure and use
% version 4.0 (2003-05-27) and later if using hyperref.sty. cite.sty does
% not currently provide for hyperlinked citations.
% The latest version can be obtained at:
% http://www.ctan.org/tex-archive/macros/latex/contrib/cite/
% The documentation is contained in the cite.sty file itself.






% *** GRAPHICS RELATED PACKAGES ***
%
\ifCLASSINFOpdf
   \usepackage[pdftex]{graphicx}
  % declare the path(s) where your graphic files are
   \graphicspath{{../pdf/}{../jpeg/}{../png/}}
  % and their extensions so you won't have to specify these with
  % every instance of \includegraphics
  \DeclareGraphicsExtensions{.pdf,.jpeg,.png}
  \usepackage[justification=centering]{caption}
\else
  % or other class option (dvipsone, dvipdf, if not using dvips). graphicx
  % will default to the driver specified in the system graphics.cfg if no
  % driver is specified.
  \usepackage[dvips]{graphicx}
  % declare the path(s) where your graphic files are
  \graphicspath{{../eps/}}
  % and their extensions so you won't have to specify these with
  % every instance of \includegraphics
  \DeclareGraphicsExtensions{.eps}
\fi
% graphicx was written by David Carlisle and Sebastian Rahtz. It is
% required if you want graphics, photos, etc. graphicx.sty is already
% installed on most LaTeX systems. The latest version and documentation can
% be obtained at: 
% http://www.ctan.org/tex-archive/macros/latex/required/graphics/
% Another good source of documentation is "Using Imported Graphics in
% LaTeX2e" by Keith Reckdahl which can be found as epslatex.ps or
% epslatex.pdf at: http://www.ctan.org/tex-archive/info/
%
% latex, and pdflatex in dvi mode, support graphics in encapsulated
% postscript (.eps) format. pdflatex in pdf mode supports graphics
% in .pdf, .jpeg, .png and .mps (metapost) formats. Users should ensure
% that all non-photo figures use a vector format (.eps, .pdf, .mps) and
% not a bitmapped formats (.jpeg, .png). IEEE frowns on bitmapped formats
% which can result in "jaggedy"/blurry rendering of lines and letters as
% well as large increases in file sizes.
%
% You can find documentation about the pdfTeX application at:
% http://www.tug.org/applications/pdftex





% *** MATH PACKAGES ***
%
\usepackage[cmex10]{amsmath}
% A popular package from the American Mathematical Society that provides
% many useful and powerful commands for dealing with mathematics. If using
% it, be sure to load this package with the cmex10 option to ensure that
% only type 1 fonts will utilized at all point sizes. Without this option,
% it is possible that some math symbols, particularly those within
% footnotes, will be rendered in bitmap form which will result in a
% document that can not be IEEE Xplore compliant!
%
% Also, note that the amsmath package sets \interdisplaylinepenalty to 10000
% thus preventing page breaks from occurring within multiline equations. Use:
%\interdisplaylinepenalty=2500
% after loading amsmath to restore such page breaks as IEEEtran.cls normally
% does. amsmath.sty is already installed on most LaTeX systems. The latest
% version and documentation can be obtained at:
% http://www.ctan.org/tex-archive/macros/latex/required/amslatex/math/





% *** SPECIALIZED LIST PACKAGES ***
%
\usepackage{algorithm}
\usepackage{algorithmic}
% algorithmic.sty was written by Peter Williams and Rogerio Brito.
% This package provides an algorithmic environment fo describing algorithms.
% You can use the algorithmic environment in-text or within a figure
% environment to provide for a floating algorithm. Do NOT use the algorithm
% floating environment provided by algorithm.sty (by the same authors) or
% algorithm2e.sty (by Christophe Fiorio) as IEEE does not use dedicated
% algorithm float types and packages that provide these will not provide
% correct IEEE style captions. The latest version and documentation of
% algorithmic.sty can be obtained at:
% http://www.ctan.org/tex-archive/macros/latex/contrib/algorithms/
% There is also a support site at:
% http://algorithms.berlios.de/index.html
% Also of interest may be the (relatively newer and more customizable)
% algorithmicx.sty package by Szasz Janos:
% http://www.ctan.org/tex-archive/macros/latex/contrib/algorithmicx/




% *** ALIGNMENT PACKAGES ***
%
%\usepackage{array}
% Frank Mittelbach's and David Carlisle's array.sty patches and improves
% the standard LaTeX2e array and tabular environments to provide better
% appearance and additional user controls. As the default LaTeX2e table
% generation code is lacking to the point of almost being broken with
% respect to the quality of the end results, all users are strongly
% advised to use an enhanced (at the very least that provided by array.sty)
% set of table tools. array.sty is already installed on most systems. The
% latest version and documentation can be obtained at:
% http://www.ctan.org/tex-archive/macros/latex/required/tools/


%\usepackage{mdwmath}
%\usepackage{mdwtab}
% Also highly recommended is Mark Wooding's extremely powerful MDW tools,
% especially mdwmath.sty and mdwtab.sty which are used to format equations
% and tables, respectively. The MDWtools set is already installed on most
% LaTeX systems. The lastest version and documentation is available at:
% http://www.ctan.org/tex-archive/macros/latex/contrib/mdwtools/


% IEEEtran contains the IEEEeqnarray family of commands that can be used to
% generate multiline equations as well as matrices, tables, etc., of high
% quality.


%\usepackage{eqparbox}
% Also of notable interest is Scott Pakin's eqparbox package for creating
% (automatically sized) equal width boxes - aka "natural width parboxes".
% Available at:
% http://www.ctan.org/tex-archive/macros/latex/contrib/eqparbox/





% *** SUBFIGURE PACKAGES ***
%\usepackage[tight,footnotesize]{subfigure}
% subfigure.sty was written by Steven Douglas Cochran. This package makes it
% easy to put subfigures in your figures. e.g., "Figure 1a and 1b". For IEEE
% work, it is a good idea to load it with the tight package option to reduce
% the amount of white space around the subfigures. subfigure.sty is already
% installed on most LaTeX systems. The latest version and documentation can
% be obtained at:
% http://www.ctan.org/tex-archive/obsolete/macros/latex/contrib/subfigure/
% subfigure.sty has been superceeded by subfig.sty.



%\usepackage[caption=false]{caption}
%\usepackage[font=footnotesize]{subfig}
% subfig.sty, also written by Steven Douglas Cochran, is the modern
% replacement for subfigure.sty. However, subfig.sty requires and
% automatically loads Axel Sommerfeldt's caption.sty which will override
% IEEEtran.cls handling of captions and this will result in nonIEEE style
% figure/table captions. To prevent this problem, be sure and preload
% caption.sty with its "caption=false" package option. This is will preserve
% IEEEtran.cls handing of captions. Version 1.3 (2005/06/28) and later 
% (recommended due to many improvements over 1.2) of subfig.sty supports
% the caption=false option directly:
%\usepackage[caption=false,font=footnotesize]{subfig}
%
% The latest version and documentation can be obtained at:
% http://www.ctan.org/tex-archive/macros/latex/contrib/subfig/
% The latest version and documentation of caption.sty can be obtained at:
% http://www.ctan.org/tex-archive/macros/latex/contrib/caption/




% *** FLOAT PACKAGES ***
%
%\usepackage{fixltx2e}
% fixltx2e, the successor to the earlier fix2col.sty, was written by
% Frank Mittelbach and David Carlisle. This package corrects a few problems
% in the LaTeX2e kernel, the most notable of which is that in current
% LaTeX2e releases, the ordering of single and double column floats is not
% guaranteed to be preserved. Thus, an unpatched LaTeX2e can allow a
% single column figure to be placed prior to an earlier double column
% figure. The latest version and documentation can be found at:
% http://www.ctan.org/tex-archive/macros/latex/base/



%\usepackage{stfloats}
% stfloats.sty was written by Sigitas Tolusis. This package gives LaTeX2e
% the ability to do double column floats at the bottom of the page as well
% as the top. (e.g., "\begin{figure*}[!b]" is not normally possible in
% LaTeX2e). It also provides a command:
%\fnbelowfloat
% to enable the placement of footnotes below bottom floats (the standard
% LaTeX2e kernel puts them above bottom floats). This is an invasive package
% which rewrites many portions of the LaTeX2e float routines. It may not work
% with other packages that modify the LaTeX2e float routines. The latest
% version and documentation can be obtained at:
% http://www.ctan.org/tex-archive/macros/latex/contrib/sttools/
% Documentation is contained in the stfloats.sty comments as well as in the
% presfull.pdf file. Do not use the stfloats baselinefloat ability as IEEE
% does not allow \baselineskip to stretch. Authors submitting work to the
% IEEE should note that IEEE rarely uses double column equations and
% that authors should try to avoid such use. Do not be tempted to use the
% cuted.sty or midfloat.sty packages (also by Sigitas Tolusis) as IEEE does
% not format its papers in such ways.





% *** PDF, URL AND HYPERLINK PACKAGES ***
%
%\usepackage{url}
% url.sty was written by Donald Arseneau. It provides better support for
% handling and breaking URLs. url.sty is already installed on most LaTeX
% systems. The latest version can be obtained at:
% http://www.ctan.org/tex-archive/macros/latex/contrib/misc/
% Read the url.sty source comments for usage information. Basically,
% \url{my_url_here}.





% *** Do not adjust lengths that control margins, column widths, etc. ***
% *** Do not use packages that alter fonts (such as pslatex).         ***
% There should be no need to do such things with IEEEtran.cls V1.6 and later.
% (Unless specifically asked to do so by the journal or conference you plan
% to submit to, of course. )


% correct bad hyphenation here
\hyphenation{op-tical net-works semi-conduc-tor}


\begin{document}
%
% paper title
% can use linebreaks \\ within to get better formatting as desired
\title{Taking One Small Step Forward:\\Finding Low-Frequentcy Items in Data Streams}


% author names and affiliations
% use a multiple column layout for up to three different
% affiliations
\author{\IEEEauthorblockN{Michael Shell}
\IEEEauthorblockA{School of Electrical and\\Computer Engineering\\
Georgia Institute of Technology\\
Atlanta, Georgia 30332--0250\\
Email: http://www.michaelshell.org/contact.html}
\and
\IEEEauthorblockN{Homer Simpson}
\IEEEauthorblockA{Twentieth Century Fox\\
Springfield, USA\\
Email: homer@thesimpsons.com}
\and
\IEEEauthorblockN{James Kirk\\ and Montgomery Scott}
\IEEEauthorblockA{Starfleet Academy\\
San Francisco, California 96678-2391\\
Telephone: (800) 555--1212\\
Fax: (888) 555--1212}}

% conference papers do not typically use \thanks and this command
% is locked out in conference mode. If really needed, such as for
% the acknowledgment of grants, issue a \IEEEoverridecommandlockouts
% after \documentclass

% for over three affiliations, or if they all won't fit within the width
% of the page, use this alternative format:
% 
%\author{\IEEEauthorblockN{Michael Shell\IEEEauthorrefmark{1},
%Homer Simpson\IEEEauthorrefmark{2},
%James Kirk\IEEEauthorrefmark{3}, 
%Montgomery Scott\IEEEauthorrefmark{3} and
%Eldon Tyrell\IEEEauthorrefmark{4}}
%\IEEEauthorblockA{\IEEEauthorrefmark{1}School of Electrical and Computer Engineering\\
%Georgia Institute of Technology,
%Atlanta, Georgia 30332--0250\\ Email: see http://www.michaelshell.org/contact.html}
%\IEEEauthorblockA{\IEEEauthorrefmark{2}Twentieth Century Fox, Springfield, USA\\
%Email: homer@thesimpsons.com}
%\IEEEauthorblockA{\IEEEauthorrefmark{3}Starfleet Academy, San Francisco, California 96678-2391\\
%Telephone: (800) 555--1212, Fax: (888) 555--1212}
%\IEEEauthorblockA{\IEEEauthorrefmark{4}Tyrell Inc., 123 Replicant Street, Los Angeles, California 90210--4321}}




% use for special paper notices
%\IEEEspecialpapernotice{(Invited Paper)}




% make the title area
\maketitle


\begin{abstract}
%\boldmath
We propose an one-pass algorithm, called BFSS, which can identify items in a data stream with frequencies less than a user specified
support. Our algorithm is simple and have provably small memory footprints related to domain of data streams. Although the output is approximate, we can guarantee no false negatives and provably few false positives. Given a little modification, algorithm BFSS can be improved to
SBFSS, which can handle low-frequency items detection over data streams form large domain with acceptable and bounded false negatives and false positives using a proper space.
\end{abstract}
% IEEEtran.cls defaults to using nonbold math in the Abstract.
% This preserves the distinction between vectors and scalars. However,
% if the conference you are submitting to favors bold math in the abstract,
% then you can use LaTeX's standard command \boldmath at the very start
% of the abstract to achieve this. Many IEEE journals/conferences frown on
% math in the abstract anyway.

% no keywords




% For peer review papers, you can put extra information on the cover
% page as needed:
% \ifCLASSOPTIONpeerreview
% \begin{center} \bfseries EDICS Category: 3-BBND \end{center}
% \fi
%
% For peerreview papers, this IEEEtran command inserts a page break and
% creates the second title. It will be ignored for other modes.
\IEEEpeerreviewmaketitle



\section{Introduction}
% no \IEEEPARstart
In many real-world applications, information such as web click data, stock ticker
data, sensor network data, phone call records, and traffic monitoring data appear in the form
of data streams. Online monitoring of data streams has emerged as an important research
undertaking. Estimating the frequency of the items on these streams is an important aggregation and summary technique for both stream mining and data management systems with a broad range of applications. A variety of algorithms have been proposed to identify frequent items from data streams, \emph{Sticky Sampling} [] and \emph{Space Saving} [] etc.\par
However, to the best of our knowledge, there are no algorithms identifying low-frequency items over data streams. Low-frequency items, similar to the defination of frequent items, are items frequencies of which are less than a specified threshold S. \par
Zipf's law refers to the fact that many types of data studied in the physical and social sciences can be approximated with a Zipfian distribution, one of a family of related discrete power law distributions. Fig.\ref{fig:sim} describes relation between frequency rank and frequency of items that follow a power law distribution over data streams, and we can find that the distinct number of low-frequency items is much larger than the distinct number of frequent items, so under the restriction of limited memory it is much more difficult to identify low-frequency items than to identify frequent items. \par
Nowadays, Low-frequency items over data streams are becoming more and more important because of the rich information they contain which can be easily understood through the observation of entropy of data streams []. We take one small step forward to maintain low-frequency items over data streams approximately and try to fill up the blank of low-frequency items mining over data streams in this paper.   
% You must have at least 2 lines in the paragreaph with the drop letter
% (should never be an issue)

\begin{figure}
	\centering
	\includegraphics[width=2.5in]{png/zipf.png}
	\caption{Rank-frequency distribution}
	\label{fig:sim}
\end{figure}


\subsection{Motivating Examples}
\subsubsection{Individual requirements mining}
In such an era that information technology develops rapidly, it is much easier for us to get access to various of knowledge and information and we need through a few mouse clicks, for example, we can shop online with the help of e-commerce site, such as Amazon and Taobao etc. We can search whatever we are interested in through search engines, such as Google and Bing etc. \par 
Our requirements are popular most times, for example, buying a regular water glass online or searching the information about a tourist attraction etc, and these popular requirements can be easily met because almost every e-commerce site sell all linds of water glasses and nearly every search engines provide links to popular tourist attractions worldwide.  \par
However, we are no longer saiesfied with responses only to popular demands. For example, it is not so easy for us to buy embroidery stitches online or search information about a nameless small village in China, because these are individual requirements, and some e-commerce sites or search engines may neglect these requirements, for example, i can't find product information about embroidery stitches at Amazon while Taobao does. \par
Yusuf Mehdi, SVP of Microsoft, once said at the meeting of Search Engine Strategies in 2010 that major reason why Bing got begind with Google is neglecting ``long tail'' of search flow items of which appear a few times or even once, and it is of great importance for them to analyze these low-frequency items. So it is individual requirements or in other words low-frequency demands, in some sense, that really make a difference rather than popular demands in areas like e-commerce and SE, and identifying them is the first step to do deep analysis on them.

\subsubsection{Data distribution estimation}

\subsection{Our Contributions}
We introduce and state the proplem of low-frequency items detection over data streams which are of great significance in areas such as individual requirements mining and data distribution estimation, and to the best of our knowledge there is no relative research up to now.\par

In this paper, we propose \emph{BFSS}, which extends the classic frequent items detection algorithm \emph{Space Saving} to maintain both frequent and low-frequency items in a data stream approximately. The basic idea of our solution is as follows: each item in a data stream is either a frequent one or a low-frequency one once the threshold $\phi[\in (0,1)]$ is set, so we can maintain low-frequency items by filtering the frequent items out. A major problem we have to deal with is to maintain an itemset items in which appear in the data stream, and this can be done approximately using a Bloom filter size of which is related to the size of domain of data streams. \emph{BFSS} gurantee no false negatives and provably few false positives using small memory footprints.\par

However, size of a Bloom filter must increase with the size of domain in order to keep a low false positive rate, and here comes a problem: what if the domain is too large to be handled by a Bloom filter in a limited space? Inspired by the solution presented in [], we propose \emph{SBFSS} which extends \emph{BFSS} to deal with data streams from large domains, and \emph{SBFSS} gurantee acceptable and bounded false negatives and false positives using a proper space.

\subsection{Roadmap}
In Section 2, we present the problem statement and some background on existing approches which deal with the problem of frequent items detection. Our solutions are presented and discussed in Section 3. In Section 4, we discuss how our algorithms can be used in practice. In Section 5, we experimentally evaluate out methods. Conclusions are given in Section 6.

\section{PRELIMINARIES}
This section presents the problem statement and some representative algorithms solvimg $\epsilon$\emph{-Deficient Frequent Elements} [sticky sampling] which will be formally defined below.Table \ref{tab:list} summarizes the major notation in this paper.

\begin{table}
	
    \caption{Major Notation Used in the Paper}
	\begin{tabular}{ll}
		\hline
		Notation  & Meaning\\ 
		\hline
		\emph{BFSS} & The name of our first algorithm\\
		\emph{SBFSS} & The name of our second algorithm\\		
		$S$ & The input data stream\\
		$A$ & The domain of $S$\\
		$M$ & The size of $A$\\
		$M'$ & The number of distinct elements in $S$\\
		$\phi$ & The user specified support threshold\\
		$D$ & The synopsis used in \emph{Space-Saving}\\
		\emph{BF}  &The Bloom filter used in \emph{BFSS}\\
		$K$ & The size of \emph{BF}\\
		$C$ & The number of counters in $D$\\
		$N$ & The number of elements in the input stream\\
	    $H$ & The number of hash functions\\
	    $h_i$ & The $ith$ hash function\\
	    $f_S(e)$ & The frequency of element $e$ in $S$\\ 
	    \emph{FPs} &The elements in $S$ with frequency no less than $\lfloor \phi N\rfloor$ wrongly output\\
	    \emph{FNs} & The elements in $S$ with frequency less than $\lfloor \phi N\rfloor$ wrongly neglected\\
	    $(e,f(e),\Delta(e))$ & The form of each counter in $D$\\ 
	    $E$ & The set of elements monitored in $D$\\
	    $min$ & The minimum value of $f(e)$ in $D$\\
        \hline
	\end{tabular}
\label{tab:list}

\end{table}

\subsection{Problem Statement}
Consider an input stream $S = e_1,e_2,..., e_N$ of current length $N$, which arrives item by item. Let each item $e_i$ belong to a universe set $A=\{a_1,a_2,...,a_M\}$ of size $M$ representing the input stream's domain. Let $f_S(a)$ denote the number of occurrences of $a$ in $S$\par

Our problem $\phi$\emph{-Bounded Low-Frequency Elements} can be stated as follows: given a data stream $S$ along with its domain $A$ and a user specified threshold $\phi[\in (0,1)]$, output the subset $I(S,\phi) \subset A$ of symbols defined as $I(S,\phi) = \{a\in A : 0 < f_S(a)\leq\lfloor \phi N\rfloor\}$.\par

At any point of time, \emph{BFSS} output a list of items with the following guarantees: 
\begin{enumerate}
\item  All elements whose true frequency is no more than $\lfloor\phi N\rfloor$ are output. There are \emph{no false negatives}.
\item  No element whose true frequency exceeds $2\lfloor\phi N\rfloor$ is output.
\end{enumerate} 

As for data streams from large domains which means $M$ is so large that \emph{BFSS} can not handle in mempry, \emph{SBFSS} output a list of items using proper space under the restriction of limited memory with acceptable false negatives and false positives, where a false positive is an item with frequency more than $\lfloor\phi N\rfloor$ wrongly output, and a false negative is an item with frequency no more than $\lfloor\phi N\rfloor$ wrongly neglected.

\subsection{Related Work}
As far as we know that there are no related algorithms addressing this problem, however, the methods we propose are based on algorithm solving $\epsilon$\emph{-Deficient Frequent Elements} which can be described as follows: given a input stream $S$ of current length $N$ and a support threshold $\phi \in (0,1)$, return the items whose frequencies are guaranteed to be no smaller than $\lfloor(\phi-\epsilon)N\rfloor$ deterministically or with a probability of at least $1-\delta$, where $\epsilon \in (0,1)$ is a user-defined error and $\delta \in (0,1)$ is a probability of failure, so we examine several algorithms solving $\epsilon$\emph{-Deficient Frequent Elements}.\par

research can be divided into two groups: \emph{counter-based} techniques and \emph{sketch-based} techniques.\par

\textbf{Counter-Based Techniques} keep an individual counter for each element in the monitored set, a subset of A. The counter of a monitored element, $e_i$, is updated when $e_i$ occurs in the stream. If there is no counter kept for the observed ID, it is either disregarded, or some algorithm-dependent action is taken.\par

Two representative algorithms \emph{Sticky Sampling} and \emph{Lossy Counting} were proposed in []. The algorithms cut the stream into rounds, and they prune some potential low-requency items at the edge of each round. Though simple and intuitive, they suffer from zeroing too many counters at rounds’ boundaries, and thus, they free space before it is really needed. In addition, answering a frequent elements query entails scanning all counters.\par

The algorithm \emph{Space-Saving} , the one we are based at, was proposed in []. The algorithm maintains a synopsis which keeps all counters in an order according to the value of each counter's monitoring frequency plus maximum possible error. For a non-monitored item, the counter with the smallest counts, \emph{min}, is assigned to monitor it, with the items monitoring frequency $f(e)$ set to 1 and its maximal possible error $\Delta(e)$ set to \emph{min}. Since $min\leq\epsilon N$ (this follows because of the choice of the number of counters), the operation amounts to replacing an old, potentially infrequent item with a new, hopefully frequent item. This strategy keeps the item information until the very end when space is absolutely needed, and it leads to the high accuracy of \emph{Space-Saving}. Experiments done in [overview] showed \emph{Space-Saving} outperforms other Counter-Based techniques in recall/precision tests.\par

\textbf{Sketch-Based Techniques} do not monitor a subset of elements, rather provide, with less stringent guarantees, frequency estimation for all elements using bitmaps of counters. Usually, each element is hashed into the space of counters using a family of hash functions, and the hashed-to counters are updated for every hit of this element. Those “representative” counters are then queried for the element frequency with less accuracy, due to hashing collisions.\par

The \emph{Count-Min Sketch} algorithm of Cormode and Muthukrishnan [Count-Min] maintains an array of $d\times w$ counters, and pairwise independent hash functions $h_j$ map items onto $[w]$ for each row. Each update is mapped onto $d$ entries in the array, each of which is incremented. The Markov inequality is used to show that the estimate for each $j$ overestimates by less than $n/w$, and repeating $d$ times reduces the probability of error exponentially.\par

The \emph{hCount} algorithm was proposed in []. The data structure and algorithms used in \emph{Count-Min Sketch} and \emph{hCount} shared the similarity, but were simultaneously and independently investigated with different focuses.

\section{OUR ALGORITHMS}
In this section, we will discuss our approaches \emph{BFSS} and \emph{SBFSS} in detail.
\subsection{The Challenges in $\phi$-Bounded Low-Frequency Elements}
$\phi$\emph{-Bounded Low-Frequency Elements} has two main challenges due to the different features between frequent items and low-frequency elements over data streams.\par

\textbf{The long tail phenomenon}. It can be easily proved that there are at most $\lceil1/\phi\rceil$ frequent elements whose frequency is more than $\lfloor\phi N\rfloor$ in any data stream, however, there is no provable upper bound of the number of low-frequency elements whose frequency is less than $\lfloor\phi N\rfloor$. In fact, our experiments on both real and synthetic data show that low-frequency elements occupy most of the distinct elements appear in data streams, and it is almost impossible to maintain all low-frequency elements in memory. Another observation is that their frequencies are very low and close as well, and it may consume much space to separate low-frequency elements from frequent elements especially when $\phi$ is very small.\par

\textbf{The unpredictability}. A basic and common idea of \emph{Counter-Based Techniques} is to discard potential infrequent elements dynamically, and it is based on the fact that potential infrequent elements will never become frequent elements if they don't appear afterwards, however, this fact no longer applies to low-frequency elements in data streams because potential elements will possibly become low-frequency elements if they don't appear afterwards. The unpredictability of low-frequency elements makes it difficult to maintain them directly like frequent elements.

\subsection{The BFSS Algorithm}\label{sec:bfss}
In consideration of the challenges in $\phi$\emph{-Bounded Low-Frequency Elements}, we tried to solve the problem indirectly by filtering frequent items out which is the underlying idea of \emph{BFSS}.\par
Two algorithms are proposed for updating and outputting final result separately. Algorithm \ref{alg:bfss} maintains a Bloom filter $BF$ of size $K$ with $H$ uniformly independent hash functions $\{h_1(x),...,h_H(x)\}$ and a synopsis $D$ with $M$ counters. Each of these $H$ hash function maps an element from $A$ to $[0,...,K-1]$. Initially each bit of BF is set to 0 and $D$ has $M$ empty counters. Each newly arrived element in the stream is mapped to $H$ bits in BF by the $H$ hash functions and we set the $H$ bits to 1. Then if we observe an element that is monitored in $D$, we just increment $f(e)$. If we observe an element, $e_{new}$, that is not monitored in $D$, handle it depending on whether there is an empty counter in $D$. If there is one, we just allocate it to $e_{new}$ and set $f(e_{new})$ to 1 and set $\Delta(e_{new})$ to 0. If $D$ is full, we just replace the element that currently has the least hits, $min$, with $e_{new}$. Assign $f(e_{new})$ the value $min+1$ and assign $\Delta(e_{new})$ the value $min$.\par

\begin{algorithm}[h]
	\caption{BFSS Update Algorithm}
	\label{alg:bfss}
\begin{algorithmic}[1]
	\REQUIRE Stream $S$, support threshold $\phi$
	\STATE $N=0,c=0,C=1/\phi,K=\lambda,H=\mu$; \COMMENT{$N$: length of stream; $m$: the number of counters used in $D$; $C$: the maximum number of counters in $D$; $K$: size of Bloom filter; $H$: number of hash functions;The value of $\lambda$ and $\mu$ will be discussed in detail later.}
	\STATE Initially synopsis $D$ has $M$ empty counters, each with the form $(e,f(e),\Delta(e))$
	\FOR{$i=0$ to $K-1$}
	\STATE $BF[i]=0$
	\ENDFOR
	\FOR{each item $e$ of stream $S$}
	\FOR{$i=1$ to $H$}
	\STATE $BF[h_i(e)]=1$
	\ENDFOR
	\IF{$e$ is monitored in $D$}
	\STATE $f(e)=f(e)+1$;
	\ELSIF{$c<C$}
	\STATE Assign a new counter $(e,1,0)$ to it
	\STATE $c=c+1$
	\ELSE
	\STATE Let $e_m$ be the element with least hits, $min$
	\STATE Replace $e_m$ with $e$ in $D$
	\STATE $f(e)=min+1,\Delta(e)=min$
	\ENDIF
	\STATE $N=N+1$;
	\ENDFOR
\end{algorithmic}
\end{algorithm}

Algorithm \ref{alg:output} checks and outputs the element with frequency less than a user-specified threshold $\phi$. For each element in $A$, we first check whether it is in $BF$. If the element is not in $BF$, it must not be a low-frequency item because it never appeared in the stream. If the element is in $BF$ but not in $D$, it must be a low-frequency item and we output it, and we will give the reason later. If the element appears both in $BF$ and $D$, we identify the element as a low-frequency element if $f(e)<\lfloor \phi N\rfloor+\Delta(e)$ with high probability. The time requirement of Algorithm \ref{alg:output} is linear to the range of universe. It is acceptable when the frequency of the requests is not high. 

\begin{algorithm}[h]
	\caption{BFSS Query Algorithm}
	\label{alg:output}
	\begin{algorithmic}[1]
		\REQUIRE \emph{BF}, $D$, $\phi$, $A$, $N$ \COMMENT{$A$: domain of stream}
		\ENSURE low-frequency elements with threshold $\phi$
		\STATE $V=|A|$, $flag=true$; \COMMENT{$V$: size of $A$; \emph{flag}: indicate whether an element is in \emph{BF} or not}
		\FOR{$i=0$ to $V-1$}
		\STATE $flag=true$
		\FOR{$j=1$ to $H$}
		\IF{$BF[h_j(A[i])]==0$}
		\STATE $flag=false$
		\STATE break;
	    \ENDIF
	    \ENDFOR
	    \IF{$flag==true$}
	    \IF{$A[i]$ is monitored in $D$}
	    \IF{$f(A[i])\leq\lfloor \phi N\rfloor+\Delta(A[i])$}
	    \STATE output $A[i]$ as a low-frequency element
	    \ENDIF
	    \ELSE
	    \STATE output $A[i]$ as a low-frequency element
	    \ENDIF
	    \ENDIF
	    \ENDFOR
   \end{algorithmic}
\end{algorithm}

\subsection{Analysis of BFSS}
In this section, we present a theoretical analysis of \emph{BFSS} described in Section \ref{sec:bfss}. We analyze the \emph{FNs} and \emph{FPs}, space complexity, and time complexity.\par


\subsubsection{\textbf{Analysis of FNs}}
In this section, we will prove that there are no \emph{FNs} in the output of \emph{BFSS}. The proof is based on Lemma \ref{lem:1} to Lemma \ref{lem:5}, and the detailed proof of Lemma \ref{lem:1} to Lemma \ref{lem:3} can be found in [].\par

\newtheorem{lemma}{Lemma}
\begin{lemma}\label{lem:1}
  $N=\sum_{\forall i|e_i \in E}(f(e_i))$
\end{lemma}

\begin{IEEEproof}
Every hit in $S$ increments only one counter by 1 among the $M$ counters which can be easily proved when $D$ has empty counters. It is true even when a replacement happens, i.e., the observed element $e$ was not previously monitored and $D$ has no empty counters, and it replaces another element $e_m$. This is because we add $f(e_m)$ to $f(e)$ and increment $f(e)$ by 1. Therefore, at any time, the sum of all counters is equal to the length of the stream observed so far.
\end{IEEEproof}


\begin{lemma} \label{lem:2}
Among all counters in $D$, the minimum counter value, $min$, is no greater than $\lfloor\frac{N}{m}\rfloor$.
\end{lemma}

\begin{IEEEproof}
Lemma \ref{lem:1} can be written as:
\begin{equation}\label{equa:1}
	min=\frac{N-\sum_{\forall i|e_i \in E}(f(e_i)-min)}{m}
\end{equation}
\indent All the items in the summation of Equation \ref{equa:1} are non negative because all counters are no smaller than $min$, hence $min\leq\lfloor \frac{N}{m}\rfloor$.
\end{IEEEproof}

\begin{lemma}\label{lem:3}
	For any element $e \in E$, $0 \leq \Delta(e) \leq min$. 
\end{lemma}

\begin{IEEEproof}
From Algorithm \ref{alg:bfss}, $\Delta(e)$ is non-negative because any observed element is always given the benefit of doubt. $\Delta(e)$ is always assigned the value of the minimum counter at the time $e$ started being observed. Since the value of the minimum counter monotonically increases over time until it reaches the current $min$, then for all monitored elements $\Delta(e) \leq min$.
\end{IEEEproof}

\begin{lemma}\label{lem:4}
	An element $e$ with $f_S(e)> min$, must be monitored in $D$.
\end{lemma}

\begin{IEEEproof}
The proof is given in [], however, we think the proof is not very rigorous because it is based on the fact that the true frequency of $e$ must be no more than its estimated frequency, i.e. $f_S(e)\leq f(e)$, but this fact should be based on Lemma \ref{lem:4}.\par
My proof is also by contradiction. Assume $e\notin E$. Then, it was evicted previously, and we assume that it had been monitored in $i(>0)$ time slots, and $e$ appeared $n_j(0<j\leq i)$ times in the $j$th time slot. $n_j$ satisfies:
\begin{equation}\label{equa:2}
\sum_{j=1}^{i}n_j=f_S(e)
\end{equation}\par
We assume that $\Delta(e_j)(>0)$ denotes the error estimation assigned to $e$ at the start of the $j$th time slot, and we have the following inequality because the minimum counter value increases monotonically:
\setlength{\arraycolsep}{0.0em}
\begin{eqnarray}\label{eqnarray:1}
\Delta(e_1)&+&n_1\leq\Delta(e_2)\notag\\
\Delta(e_2)&+&n_2\leq\Delta(e_3)\notag\\
&...&\\
\Delta(e_{i-1})&+&n_{i-1}\leq\Delta(e_i)\notag\\
\Delta(e_i)&+&n_i\leq min \notag
\end{eqnarray}\par
\setlength{\arraycolsep}{5pt}
After adding up the left and right sides of inequality group \ref{eqnarray:1}, we can get:
\begin{equation}\label{equa:3}
\Delta(e_1)+\sum_{j=1}^{i}n_j\leq min
\end{equation}\par
We can get $f_S(e)\leq min$ from equation \ref{equa:2} and inequation \ref{equa:3}. This contradicts the condition $f_S(e)>min$.\par
In fact, Lemma \ref{lem:4} can also be stated as: For any element $e\notin E$, $f_S(e)\leq min$. 
\end{IEEEproof}

\begin{lemma}\label{lem:5}
  For any element $e\in E$, $f(e)-\Delta(e)\leq f_S(e)\leq f(e)\leq f(e)-\Delta(e)+min$
\end{lemma}

\begin{IEEEproof}
From Lemma \ref{lem:3}, we can easily get $f(e)\leq f(e)-\Delta(e)+min$. $f(e)-\Delta(e)$ is the true frequency of $e$ since it was lastly observed, so $f(e)-\Delta(e)\leq f_S(e)$. From Lemma \ref{lem:4}, we can find that $\Delta(e)$ over estimated the frequency of $e$ before it was observed ,and it clearly indicates $f_S(e)\leq f(e)$. From the above, we can prove Lemma \ref{lem:5}.
\end{IEEEproof}

\newtheorem{theorem}{Theorem}
\begin{theorem}\label{thm:1}
	There are no \emph{FNs} in the output of \emph{BFSS}.
\end{theorem}

\begin{IEEEproof}
We only hava to prove that the elements we don't output contain no low-frequency elements. From Algorithm 1, only two kinds of elements are not ouput: i) elements filtered out by the \emph{BF}. ii) elements monitored in $D$ with $f(e)-\Delta(e)>\lfloor \phi N\rfloor$. The first kind of elements are obviously not low-frequency elements because they never appeared in $S$. From Lemma \ref{lem:5}, we know that the elements monitored in $D$ with $f(e)-\Delta(e)>\lfloor \phi N\rfloor$, must satisfy $f_S(e)>\lfloor \phi N\rfloor$. 
\end{IEEEproof}

\subsubsection{\textbf{Anysis of FPs}}
In this section, we will give a theoretically strict bound of expectation of number of \emph{FPs} in output of \emph{BFSS} regardless of data distribution of $S$, and a tighter bound can be derived for data stream under power law distribution.

\begin{lemma}\label{lem:6}
	 Any element $e$ with $f_S(e)>2\lfloor \phi N\rfloor$, must not be output.
\end{lemma}

\begin{IEEEproof}
From Lemma \ref{lem:4}, we know that any element $e$ with $f_S(e)>2\lfloor \phi N\rfloor$ must be monitored in $D$, i.e. $e\in E$. Then from Lemma \ref{lem:5}, we can get that elements monitored in $D$ must satisfy $f_S(e)\leq f(e)$, and that is to say any element $e$ monitored in $D$ with $f_S(e)>2\lfloor \phi N\rfloor$, must satisfy $f(e)>2\lfloor \phi N\rfloor$. At last from Algorithm \ref{alg:bfss}, we can prove that any element $e$ with $f_S(e)>2\lfloor \phi N\rfloor$, must not be output.
\end{IEEEproof}

\begin{lemma}\label{lem:7}
The probability of a false positive of \emph{BF} is no more than
\begin{equation}
(1-(1-\frac{1}{K})^{HM})^H
\end{equation}
\end{lemma}

\begin{IEEEproof}
First, a false positive of \emph{BF} means an element in $A$ not appearing in $S$ but not filtered out by \emph{BF}. Observe that after inserting $M$ keys into a table of size $K$, the probability that a particular bit is still 0 is exactly
\begin{equation} 
(1-\frac{1}{K})^{HM}
\end{equation}
\indent Hence the probability of a false positive in this situation is $(1-(1-\frac{1}{K})^{HM})^H$. However, we know that $M$ denotes the size of $A$ which is the domain of $S$, so the number of distinct elements in $S$ must be no more than $M$, i.e. $M'\leq M$, and further we have $(1-(1-\frac{1}{K})^{HM'})^H\leq (1-(1-\frac{1}{K})^{HM})^H$.
\end{IEEEproof}

\begin{theorem}\label{thm:2}
Assuming no specific data distribution, the expectation of the number of \emph{FPs} in the output of \emph{BFSS}, denoted as $E(\#FPs)$, satisfies:
\begin{equation}\label{eq:7}
E(\#FPs)<M(1-(1-\frac{1}{K})^{HM})^H + \frac{1}{\phi}
\end{equation}
\end{theorem}

\begin{IEEEproof}
From Algorithm \ref{alg:bfss}, we can observe that two kinds of elements contribute to \emph{FPs}: i) the false positives of \emph{BF}; ii) the elements with $f_S(e)>\lfloor \phi N\rfloor$ wrongly output.\par
For the first case, we define the independent 0-1 random variables $x_i(1\leq i\leq M)$ for each element in $A$, and the value of $x_i$ depends on whether $a_i$ appeared in $S$ or not. If $a_i$ appeared in $S$, then $x_i=0$; If not, then with a probability of $(1-(1-\frac{1}{K})^{HM'})^H$, $x_i=1$; From the defination of $x_i$, we can find that the expectation of the number of the false positives of \emph{BF} equals $E(\sum_{i=1}^{M}x_i)$, i.e. $(M-M')(1-(1-\frac{1}{K})^{HM'})^H$.\par 
For the second case, we know from Lemma \ref{lem:6} that elements with $f_S(e)>2\lfloor \phi N\rfloor$ must not be not output, so only elements with $\lfloor \phi N\rfloor<f_S(e)\leq 2\lfloor \phi N\rfloor$ are likely to be output, and these elements must be monitored in $D$ which can be directly derived from Lemma \ref{lem:4}, so the maximum number of these elements is $\frac{1}{\phi}$. From above, due to the linear properties of expectation, we can get:
\begin{equation}\label{eq:8}
E(\#FPs)\leq (M-M')(1-(1-\frac{1}{K})^{HM'})^H+\frac{1}{\phi}\\
\end{equation}
\indent In addition, $M'\leq M$, so inequation \ref{eq:7} can be easily derived from inequation \ref{eq:8}.
\end{IEEEproof}

\begin{theorem}\label{thm:3}
	\begin{equation}\label{eq:9}
	E(\#FPs)<M(1-(1-\frac{1}{K})^{HM})^H + \frac{1}{\phi}
	\end{equation}
\end{theorem}

\begin{IEEEproof}

\end{IEEEproof}

\subsubsection{}
Subsubsection text here.


% An example of a floating figure using the graphicx package.
% Note that \label must occur AFTER (or within) \caption.
% For figures, \caption should occur after the \includegraphics.
% Note that IEEEtran v1.7 and later has special internal code that
% is designed to preserve the operation of \label within \caption
% even when the captionsoff option is in effect. However, because
% of issues like this, it may be the safest practice to put all your
% \label just after \caption rather than within \caption{}.
%
% Reminder: the "draftcls" or "draftclsnofoot", not "draft", class
% option should be used if it is desired that the figures are to be
% displayed while in draft mode.
%
%\begin{figure}[!t]
%\centering
%\includegraphics[width=2.5in]{myfigure}
% where an .eps filename suffix will be assumed under latex, 
% and a .pdf suffix will be assumed for pdflatex; or what has been declared
% via \DeclareGraphicsExtensions.
%\caption{Simulation Results}
%\label{fig_sim}
%\end{figure}

% Note that IEEE typically puts floats only at the top, even when this
% results in a large percentage of a column being occupied by floats.


% An example of a double column floating figure using two subfigures.
% (The subfig.sty package must be loaded for this to work.)
% The subfigure \label commands are set within each subfloat command, the
% \label for the overall figure must come after \caption.
% \hfil must be used as a separator to get equal spacing.
% The subfigure.sty package works much the same way, except \subfigure is
% used instead of \subfloat.
%
%\begin{figure*}[!t]
%\centerline{\subfloat[Case I]\includegraphics[width=2.5in]{subfigcase1}%
%\label{fig_first_case}}
%\hfil
%\subfloat[Case II]{\includegraphics[width=2.5in]{subfigcase2}%
%\label{fig_second_case}}}
%\caption{Simulation results}
%\label{fig_sim}
%\end{figure*}
%
% Note that often IEEE papers with subfigures do not employ subfigure
% captions (using the optional argument to \subfloat), but instead will
% reference/describe all of them (a), (b), etc., within the main caption.


% An example of a floating table. Note that, for IEEE style tables, the 
% \caption command should come BEFORE the table. Table text will default to
% \footnotesize as IEEE normally uses this smaller font for tables.
% The \label must come after \caption as always.
%
%\begin{table}[!t]
%% increase table row spacing, adjust to taste
%\renewcommand{\arraystretch}{1.3}
% if using array.sty, it might be a good idea to tweak the value of
% \extrarowheight as needed to properly center the text within the cells
%\caption{An Example of a Table}
%\label{table_example}
%\centering
%% Some packages, such as MDW tools, offer better commands for making tables
%% than the plain LaTeX2e tabular which is used here.
%\begin{tabular}{|c||c|}
%\hline
%One & Two\\
%\hline
%Three & Four\\
%\hline
%\end{tabular}
%\end{table}


% Note that IEEE does not put floats in the very first column - or typically
% anywhere on the first page for that matter. Also, in-text middle ("here")
% positioning is not used. Most IEEE journals/conferences use top floats
% exclusively. Note that, LaTeX2e, unlike IEEE journals/conferences, places
% footnotes above bottom floats. This can be corrected via the \fnbelowfloat
% command of the stfloats package.



\section{Conclusion}
The conclusion goes here.




% conference papers do not normally have an appendix


% use section* for acknowledgement
\section*{Acknowledgment}


The authors would like to thank...





% trigger a \newpage just before the given reference
% number - used to balance the columns on the last page
% adjust value as needed - may need to be readjusted if
% the document is modified later
%\IEEEtriggeratref{8}
% The "triggered" command can be changed if desired:
%\IEEEtriggercmd{\enlargethispage{-5in}}

% references section

% can use a bibliography generated by BibTeX as a .bbl file
% BibTeX documentation can be easily obtained at:
% http://www.ctan.org/tex-archive/biblio/bibtex/contrib/doc/
% The IEEEtran BibTeX style support page is at:
% http://www.michaelshell.org/tex/ieeetran/bibtex/
%\bibliographystyle{IEEEtran}
% argument is your BibTeX string definitions and bibliography database(s)
%\bibliography{IEEEabrv,../bib/paper}
%
% <OR> manually copy in the resultant .bbl file
% set second argument of \begin to the number of references
% (used to reserve space for the reference number labels box)
\begin{thebibliography}{1}

\bibitem{IEEEhowto:kopka}
H.~Kopka and P.~W. Daly, \emph{A Guide to \LaTeX}, 3rd~ed.\hskip 1em plus
  0.5em minus 0.4em\relax Harlow, England: Addison-Wesley, 1999.

\end{thebibliography}




% that's all folks
\end{document}


